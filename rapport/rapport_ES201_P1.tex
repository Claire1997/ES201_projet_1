\documentclass[a4paper]{article}
\usepackage[usestackEOL]{stackengine}
\usepackage[12pt]{extsizes}
\usepackage{geometry}
\usepackage{pdfpages}
\usepackage[utf8]{inputenc}
\usepackage[T1]{fontenc}
\usepackage[french]{babel}
\usepackage{amsfonts}
\usepackage{amsmath}
\usepackage{mathabx}
\usepackage{amsthm}
\usepackage{amssymb}
\usepackage{graphicx}
\usepackage{fancyhdr}
\usepackage{listings}
\usepackage{url}
\usepackage{subfigure}
\usepackage{listings}


\setlength{\parindent}{0pt}

\title{\textbf{Rapport TP4 ES201 : \\ Architecture des microprocesseurs} \vspace{0.7cm}}
\date{24/02/2020}
\author{Alessandro LEONARDI, Corentin SOUBEIRAN,  \\ Iliana VINCENTI, Ruonan QIAN}
\theoremstyle{plain}

\begin{document}
   
    \begin{figure}
		\centering
		\includegraphics[width=0.5\linewidth]{logo_ENSTA}
	\end{figure} 
   
	\maketitle
	
	\thispagestyle{empty}
	\newpage
	\setcounter{page}{1}
	\tableofcontents
	\newpage


	\section{Présentation du TP}
	
    \newpage
	\section{Profiling de l’application}
		Nous pouvons compiler en utilisant la commande : \textit{sslittle-na-sstrix-gcc <source>.c -o <exe>.ss} pour produire un binaire pour le simulateur SimpleScalar.
		\subsection{Q1}
			Nous allons generer le pourcentage de chaque classe d'instructions de ces applications en utilisant la commande : \textit{sim-profile -iclass <exe>.ss <input>}.

			\begin{table}[!htbp]
			\centering
			\begin{tabular}{|c|c|c|}
			\hline
			\multicolumn{3}{|c|}{dijkstra\_small.ss input.dat}\\
			\hline
			Classes des instructions&Nombre d’instructions&Pourcentage\\
			\hline
			load& 				26485327&	28.78\\
			\hline
			store& 				6383433&	6.94\\
			\hline
			uncond branch& 		5382775&	5.85\\
			\hline
			cond branch& 		9396219& 	10.21\\
			\hline
			int computation& 	44369706& 	48.22\\
			\hline
			fp computation&		0& 			0.00\\
			\hline
			trap& 				239& 		0.00\\
			\hline
			\end{tabular}
			\end{table}
		

			\begin{table}[!htbp]
			\centering
			\begin{tabular}{|c|c|c|}
			\hline
			\multicolumn{3}{|c|}{bf.ss input\_small.asc}\\
			\hline
			Classes des instructions&Nombre d’instructions&Pourcentage\\
			\hline
			load& 				2623& 		8.12\\
			\hline
			store& 				3501& 		45.61\\
			\hline
			uncond branch& 		167&		2.18\\
			\hline
			cond branch& 		873&		11.37\\
			\hline
			int computation& 	2504&		32.62\\
			\hline
			fp computation&		0& 			8	0.100.00\\
			\hline
			trap& 				8&			0.10\\
			\hline
			\end{tabular}
			\end{table}

		\subsection{Q2}
		

		\subsection{Q3}
		

	
	\newpage
	\section{Evaluation des performances}
	    \subsection{Q4}
	    

	    \subsection{Q5}
	    	
	
	    
	 \newpage
	 \section{Efficacité surfacique} 
	    \subsection{Q6}
	   

	    \subsection{Q7}
	   

 		\subsection{Q8}


 		\subsection{Q9}
		
	  
    \newpage
	\section{Efficacité énergétique}
		\subsection{Q10}


		\subsection{Q11}


	\newpage
	\section{Architecture système big.LITTLE}
		\subsection{Q12}


	\newpage
	\section{Faculatif}
		\subsection{Q13}


		\subsection{Q14}


	    
\end{document}